\section{Design}
\label{sec:design}
To create ClouDJ, we implemented the design described 
in subsections \ref{sec:frontend} and \ref{sec:roles} 
by integrating Google App Engine. 
Figure \ref{fig:arch} depicts the overall system.

\begin{figure}[h]
% EXAMPLE POSSIBLE USAGES, DEPENDING ON PICTURE FILE:
%\includegraphics{myfig.pdf}
%\includegraphics[width=60mm]{myfig.png}
%\includegraphics[height=60mm]{myfig.jpg}
%\includegraphics[scale=0.75]{myfig.pdf}
%\includegraphics[angle=45,width=52mm]{myfig.jpg}
\caption{The system architecture.}
\label{fig:arch}
\end{figure}

\subsection{Roles}
\label{sec:roles}
There are three major roles in our system: 
the master handler, session handler, and client. Clients own 
and store music, can share their music with others 
and can listen to music shared by others. The master handler provides 
access control, starts sessions for users who want to host, and
acts as a liaison between the clients and session servers. The session 
handler is in charge of handling synchronization messages between clients
and serving content.

\subsection{Front End}
\label{sec:frontend}
A session is an abstraction in which multiple users 
may listen to one song that is hosted by a single user. 
Our front end application allows a user to either 
create a session and become a host, or join an existing 
session and become a listener. When acting as a host, 
a user may add or remove songs from the session playlist, 
play or pause the current song, and skip to the next song. 
When acting as a listener, the user has 
no control over what song is being played. Users can 
see all sessions that they are able to join. Only 
members of a user's access control list (ACL) may 
see or join any session in which that user is a host.

\subsection{Back End}
\label{sec:backend}
The backend infrastructure is more complicated than 
the frontend. The master server keeps track of users 
currently online, user ACLs, and user membership lists 
(ACLs it is a member of). It also is responsible for 
maintaining a session table, a table that maps host 
users to sessions (this is a one-to-one mapping). 
When a client logs on, the master server informs each 
session associated to a host on the client's 
membership list that this client is a potential listener.

The session handler is the workhorse of the system. It 
services requests for sessions it is in charge of by 
routing data from the host client to relevant clients 
(listeners). It maintains the list of listeners and 
potential listeners. A potential listener is a client 
who could listen to this session, but is currently not. 
In other words, these are the clients on the host client's 
ACL that are logged in but not listening to this 
session. Session servers also take care of session 
cleanup when a session ends (or fails). 

Finally, the client exists on the user machine 
has access to its user's music and playlists and 
also keeps track of data such as the user's current 
session and the user's potential sessions (sessions 
this user can access). For the current session, the 
client keeps track of the current session key and relevant song information. 
For the potential sessions, the client keeps the host, 
session server address, and currently playing song.

Sessions are created when a client contacts the master 
handler about hosting a new session. The master handler 
then creates a new session in memory and responds to the client, 
which updates its current session information. 
Clients may join sessions by contacting the master 
handler, which updates the session information in memory
and notifies everyone in the session. Notice that after a session 
is established, the client no longer communicates 
with the master handler and the master handles 
no music data. 
