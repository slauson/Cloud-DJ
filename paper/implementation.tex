\section{Implementation}
\label{sec:implementation}

\subsection{Client and Server Communication}
Our client is written in JavaScript and uses SoundManager, a
JavaScript audio player API. In contrast, our server code is
written in Python 2.7. Both use the Google App Engine API. 

When the client wants to contact the server, it issues an HTTP
request, which is handled by the appropriate Python script handler.
Handlers in Google App Engine can only serve requests within a
time limit, which is why we chose to store session data within
the Datastore.

In contrast, when the server wants to contact a client, is uses 
the Channel service. Channels are persistent connections that 
allow updates to propagate to clients without the use of polling.
This is the mechanism with which we send updates to participants
of a session. A channel is created as follows: 
\begin{enumerate}
  \item The client contacts the server
  \item The server generates a channel ID and sends back a token
  \item The client connects to the channel using the token
\end{enumerate}
The client receives updates by listening on the channel by opening
a socket and does appropriate actions based on the message received.
Updates to the channel are made via HTTP requests from clients. When
a client disconnects from the channel, a message containing the 
corresponding channel ID is sent to the server, and appropriate
cleanup is performed.	

\subsection{Execution Flow}
% Execution flow

\subsection{Session Synchronization}
In order to synchronize playback among different clients, we use 
control messages, which are propagated via channels as described
above. Only hosts may issue commands such as play, pause, and 
next for the session. The ``play'' and ``pause'' commands are sent
only when the host clicks on the pause and play buttons, and the 
``next'' command is sent when the host switches songs or if the
host clicks on the ``next'' button. We separate the problem of synchronization 
into two parts: joining an existing session and updating the
current session. In both cases we utilize a timestamp, which is 
updated each time a play or next command is sent by the host client
and is sent with the control message. 

As seen in Figure~\ref{fig:syncJoin} when a user joins a session, 
the server sends the user the elapsed time since the last stored timestamp. 
The user then start the song with the value returned by the server as 
the offset. If the song is playing when the user joins the session, 
it will start playing at the specified offset. If the song is paused when 
the user joins the session, the retrieved offset will not be the correct 
elapsed time; however, this is corrected when the host sends a ``play'' 
message to the server, which stores another timestamp and propagates
the correct offset to all session listeners.

This timestamping synchronization is used mainly when a user joins a session.
In the other case where participants update their current session upon 
recieving a control message, the offset can usually be inferred based on the local state. 
For instance, if the user receives a ``next'' message, then they
start the next song with an offset of 0. If a user receives a ``pause" 
message, they stop the current song and keep the elapsed time as the offset.
If the user receives a ``play'' message they start the current song at the 
local offset, which was set when the ``pause'' message was received. The exception to this 
is if the current paused song is at the beginning, indicating
that this user has joined the session while the song was paused or that this user is
ahead of the host and is waiting for a control message from the host. 
In this case the user adjusts the current song offset with what 
they recieve from the server. Conversely, if the listener lags behind the host, then they will
receive a ``next'' message before the end of the song and start the next song. Figure~\ref{fig:syncUsers}
illustrates the timeline in the cases where a listener is behind the host and where a listener
is ahead of the host.

% TODO: Add figure here describing the synchronization process
% Joining a session
% Current session
\begin{figure}[h]
% EXAMPLE POSSIBLE USAGES, DEPENDING ON PICTURE FILE:
%\includegraphics{myfig.pdf}
%\includegraphics[width=60mm]{myfig.png}
%\includegraphics[height=60mm]{myfig.jpg}
%\includegraphics[scale=0.75]{myfig.pdf}
%\includegraphics[angle=45,width=52mm]{myfig.jpg}
	\centering
	\begin{subfigure}
		\centering
		\label{fig:syncJoin}
	\end{subfigure}
	
	\begin{subfigure}
		\centering
		\label{fig:syncUsers}
	\end{subfigure}
	\caption{The synchronization timeline for a) joining a session and b) updating a session}
\end{figure}


\subsection{Design Considerations}
	Performance:
		buffered songs on client
	Scalability:
		incremental messages: 
			listeners
			only on song transitions or join of session
	Availability: (failure tolerance)
		- Session info stored on Datastore
		- As long as google app engine up, our app is up